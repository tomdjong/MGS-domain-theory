\documentclass[a4paper, oneside, 12pt, openany]{book}
\pdfoutput=1

\usepackage{packages}
\usepackage{macros}

% Label tables just like equations, theorems, definitions, etc.
%
% NB: This can be confusing if LaTeX does not place the table at the point of
% writing (e.g. for lack of space)!
\numberwithin{equation}{chapter}
\numberwithin{table}{chapter}
\makeatletter
\let\c@equation\c@table
\makeatother

% Setting up the coloured environments
%
\newbool{shade-envs}
% This can be used to toggle the coloured environments on or off.
\setboolean{shade-envs}{true}

%%
\ifthenelse{\boolean{shade-envs}}{%
  % Colours are as in Andrej Bauer's notes on realizability:
  % https://github.com/andrejbauer/notes-on-realizability
  \colorlet{ShadeOfPurple}{blue!5!white}
  \colorlet{ShadeOfYellow}{yellow!5!white}
  \colorlet{ShadeOfGreen} {green!5!white}
  \colorlet{ShadeOfBrown} {brown!10!white}
  % But we also shade proofs
  \colorlet{ShadeOfGray}  {gray!10!white}
  % For exercises
  \colorlet{ShadeOfRed}   {red!10!white}
}
% If we don't want to have shaded environments, then we use a closing symbol
% \lozenge to mark the end of remarks, definitions and examples.
{%
  \declaretheoremstyle[
      spaceabove=6pt,
      spacebelow=6pt,
      bodyfont=\normalfont,
      qed=\(\lozenge\)
  ]{definitionwithbox}
  \declaretheoremstyle[
      headfont=\itshape,
      bodyfont=\normalfont,
      qed=\(\lozenge\)
      ]{remarkwithbox}
}

\ifthenelse{\boolean{shade-envs}}{%
  \declaretheorem[sibling=equation]{theorem}
  \declaretheorem[sibling=theorem]{lemma,proposition,corollary}
  \declaretheorem[sibling=theorem,style=definition]{definition}
  \declaretheorem[sibling=theorem,style=definition]{example}
  \declaretheorem[sibling=theorem,style=remark]{remark}
  \declaretheorem[sibling=theorem,style=remark]{exercise}
  % Now we set the shading using the tcolorbox package.
  %
  % The related thmtools' option "shaded" and the package mdframed seem to have
  % issues: the former does not allow for page breaks in shaded environments and
  % the latter puts double spacing between two shaded environments.
  \tcbset{shadedenv/.style={
      colback={#1},
      frame hidden,
      enhanced,
      breakable,
      boxsep=0pt,
      left=2mm,
      right=2mm,
      % LaTeX thinks this is too wide (as becomes clear from the many "Overfull
      % \hbox" warnings, but optically it looks spot on.
      add to width=1.1mm,
      enlarge left by=-0.6mm}
  }
  %
  \tcolorboxenvironment{theorem}    {shadedenv={ShadeOfPurple}}
  \tcolorboxenvironment{lemma}      {shadedenv={ShadeOfPurple}}
  \tcolorboxenvironment{proposition}{shadedenv={ShadeOfPurple}}
  \tcolorboxenvironment{corollary}  {shadedenv={ShadeOfPurple}}
  \tcolorboxenvironment{definition} {shadedenv={ShadeOfYellow}}
  \tcolorboxenvironment{example}    {shadedenv={ShadeOfGreen}}
  \tcolorboxenvironment{remark}     {shadedenv={ShadeOfBrown}}
  \tcolorboxenvironment{proof}      {shadedenv={ShadeOfGray}}
  \tcolorboxenvironment{exercise}   {shadedenv={ShadeOfRed}}
}{% Use closing symbols if we don't have colours
  \declaretheorem[sibling=equation]{theorem}
  \declaretheorem[sibling=theorem]{lemma,proposition,corollary}
  \declaretheorem[sibling=theorem,style=definitionwithbox]{definition}
  \declaretheorem[sibling=theorem,style=definitionwithbox]{example}
  \declaretheorem[sibling=theorem,style=remarkwithbox]{remark}
  \declaretheorem[sibling=theorem,style=remarkwithbox]{exercise}
  }
  \declaretheorem[sibling=theorem,style=remark,numbered=no]{claim}

% Note that proofs will still have the \qed symbol at the end, even when shaded,
% because we prefer to keep up the tradition.

\begin{document}

\frontmatter

\begin{titlepage}
\begin{center}
  \vspace*{\stretch{0.5}}

  \large % Default size for the title page

  {\Huge\textsc{Domain theory and \\ denotational semantics}\par}

  \vspace{\stretch{0.2}}

  by

  \vspace{\stretch{0.2}}

  {\huge\textsc{Tom de Jong}}

  \vspace{\stretch{0.5}}

  {\Large{Lecture notes and exercises for the\\
      \textsc{Midlands Graduate School (MGS)}}} \\
  \vspace{\stretch{0.1}}
  2--6 April 2023, Birmingham, UK

  \vspace{\stretch{1}}

  \begingroup
  \tikzset{every picture/.style={color=Gray!90!Black}}
  \begin{tikzcd}
    0 & 1 & 2 & 3 & \ldots \\
    & & \bot \ar[ull,no head] \ar[ul,no head] \ar[u,no head] \ar[ur, no head] \ar[urr, no head]
  \end{tikzcd}
  \endgroup

  \vfill

  \flushright
  {\normalsize{School of Computer Science \\
  University of Nottingham \\
  February--March 2023}}

\end{center}
\end{titlepage}
\restoregeometry%

\chapter{Abstract}

Denotational semantics aims to understand computer programs by assigning
mathematical meaning to the syntax of a programming language.
%
In this course we will study a simple functional programming language called
PCF. Notably, this language has general recursion through a fixed point
operator.
%
This means a simple denotational semantics based on sets is not
suitable. Instead, we interpret the types of PCF as certain partially ordered
sets leading to domain theory and Scott's model of PCF in particular.
%
The central theorems of soundness and computational adequacy, formulated and
proved by Plotkin, then tell us that a PCF program computes to a value if and
only if their interpretations in the model are equal.


%%% Local Variables:
%%% mode: latexmk
%%% TeX-master: "../main"
%%% End:


\setcounter{tocdepth}{2}
\tableofcontents

\mainmatter%

\chapter{Introduction}

Denotational semantics~\cite{Scott1970,ScottStrachey1971}

% Denotational semantics aims to understand computer programmes by assigning
% mathematical meaning to the syntax of a programming language.
% %
% In this course we will study a simple functional programming language called
% PCF. Notably, this language has general recursion through a fixed point
% operator.
% %
% This means a simple denotational semantics based on sets is not
% suitable. Instead, we interpret the types of PCF as certain partially ordered
% sets leading to domain theory and Scott's model of PCF in particular.
% %
% The central theorems of soundness and computational adequacy, formulated and
% proved by Plotkin, then tell us that a PCF programme computes to a value if and
% only if their interpretations in the model are equal.

Following~\cite{Escardo2007}, we might summarise as:

\begin{displayquote}
  Operational semantics is about \emph{how} we compute.

  Denotational semantics is about \emph{what} we compute.
\end{displayquote}

\section{Aims}

\section{References}

\cite{AbramskyJung1994,GierzEtAl2003,Streicher2006,Hart2020,Escardo2007}

\cite{PittsWinskellFiore2012} lecture notes

\cite{Winskel1993} textbook

\cite{Streicher2006,Gunther1992}


%%% Local Variables:
%%% mode: latexmk
%%% TeX-master: "../main"
%%% End:

\include{mainmatter/pcf}
\include{mainmatter/domains}
\chapter{The Scott model of PCF}

We started these notes by introducing PCF and its operational semantics
in~\cref{chap:PCF}. We then developed sufficient domain theory
in~\cref{chap:domains} to give a denotational semantics of PCF, where we
interpret the types of PCF as \(\omega\)-cppos and terms as
\(\omega\)-continuous maps. This chapter is devoted to precisely defining that
interpretation, known as the Scott model of PCF.

\begin{definition}[Interpretation of PCF types, \(\densem{\sigma}\)]
  We inductively assign an \(\omega\)-cppo \(\densem{\sigma}\) to each type
  \(\sigma\) of PCF:
  \[
    \densem{\pcfnat} \coloneqq \Nat_\bot
    \quad\text{and}\quad
    \densem{\sigma \pcffun \tau} \coloneqq \densem{\tau}^{\densem{\sigma}},
  \]
  % \begin{alignat*}{3}
  %   &\densem{\pcfnat} &&\coloneqq \Nat_\bot, \text{and} \\
  %   &\densem{\sigma \pcffun \tau} &&\coloneqq \densem{\tau}^{\densem{\sigma}},
  % \end{alignat*}
  where we recall \(\Nat_\bot\) from~\cref{exam:N_bot} and the exponential
  from~\cref{def:exponential}.
\end{definition}

This interpretation extends to contexts by using iterated binary products:
\begin{definition}[Interpretation of contexts, \(\densem{\Gamma}\)]
  The interpretation of a context
  \(\Gamma = [\var x_0 : \sigma_0, \dots , \var x_{n-1} :
  \sigma_{n-1}]\)
  is
  \[
    \densem{\Gamma} \coloneqq \densem{\sigma_0} \times \cdots \times
    \densem{\sigma_{n-1}}
  \]
  with the convention that the empty context is interpreted as the empty
  product: the one point \(\omega\)-cppo \(\One\) from~\cref{exam:one-point}.
\end{definition}

The interpretation of the terms of PCF proceeds by yet another inductive
definition:

\begin{definition}[Interpretation of PCF terms, \(\densem{M}\)]\label{def:interpretation}
  A term \(\Gamma \vdash M : \sigma\) of type \(\sigma\) in context~\(\Gamma\)
  will be interpreted as an \emph{\(\omega\)-continuous} function
  \[
    \densem{\Gamma} \xrightarrow{\densem{M}} \densem{\sigma}
  \]
  according to the following clauses which mirror~\cref{def:PCF-terms}:
  \begin{enumerate}[(i)]
  \item
    The interpretation of
    \[
      \def\fCenter{\ \vdash\ }
      \AxiomC{\phantom{$\fCenter$}}
      \UnaryInf$\Gamma,\var{x}:\sigma,\Delta \fCenter \var{x} : \sigma$
      \DisplayProof
    \]
    is
    \[
      \densem{\Gamma} \times \densem{\sigma} \times {\densem{\Delta}}
      \xrightarrow{\pi} \densem{\sigma},
    \]
    where \(\pi\) is a suitable composition of projections
    (recall~\cref{projections}), which is \(\omega\)-continuous, because
    \(\omega\)-continuous functions are closed under composition
    (recall~\cref{exer:category-of-cpos}).
  \item
    To interpret
    \[
      \def\fCenter{\ \vdash\ }
      \Axiom$\Gamma , \var{x} : \sigma \fCenter M : \tau$
      \UnaryInf$\Gamma \fCenter (\lambdadot{\var{x} : \sigma}{M}) : \sigma
      \pcffun \tau$ \DisplayProof
    \]
    we first remark that we have
    \[
      \densem{\Gamma} \times \densem{\sigma} \xrightarrow{\densem{M}} \densem{\tau}
    \]
    by induction hypothesis, so that we can define
    \[
      \densem{\Gamma} \xrightarrow{\densem{\lambdadot{\var x : \sigma}{M}}}
      \densem{\tau}^{\densem{\sigma}}
    \]
    as the curried (recall~\cref{exer:curry-is-continuous}) version of
    \(\densem{M}\), i.e.\ for \(\gamma \in \densem{\Gamma}\), we have
    \(\densem{\lambdadot{\var x : \sigma}{M}}(\gamma) \coloneqq x \mapsto
    \densem{M}(\gamma,x)\).
  \item
    To interpret
    \[
      \def\fCenter{\ \vdash\ }
      \Axiom$\Gamma \fCenter M : \sigma \pcffun \tau$
      \Axiom$\Gamma \fCenter N : \sigma$
      \BinaryInf$\Gamma \fCenter M \, N : \tau$
      \DisplayProof
    \]
    we first remark that we have
    \[
      \densem{\Gamma} \xrightarrow{\densem{M}} \densem{\tau}^{\densem{\sigma}}
      \quad\text{and}\quad
      \densem{\Gamma} \xrightarrow{\densem{N}} \densem{\sigma}
    \]
    by induction hypothesis, so that we can define
    \[
      \densem{\Gamma} \xrightarrow{\densem{M \, N}} \densem{\tau}
    \]
    by application: for \(\gamma \in \densem{\Gamma}\), we have
    \(\densem{M \, N}(\gamma) \coloneqq
    \densem{M}(\gamma)\pa*{\densem{N}(\gamma)}\).

    More abstractly, it is the composition of
    \[
      \densem{\Gamma} \xrightarrow{\langle \densem{M} , \densem{N} \rangle}
      \densem{\tau}^{\densem{\sigma}} \times \densem{\sigma}
      \xrightarrow{\text{evaluation}}
      \densem{\tau},
    \]
    where we recall~\cref{exer:product-induced,evaluation-is-continuous}.
  \item
    To interpret
    \[
      \def\fCenter{\ \vdash\ }
      \Axiom$\Gamma \fCenter M : \sigma \pcffun \sigma$
      \UnaryInf$\Gamma \fCenter \pcffix_\sigma(M) : \sigma$ \DisplayProof
    \]
    we first remark that we have
    \[
      \densem{\Gamma} \xrightarrow{\densem{M}} \densem{\sigma}^{\densem{\sigma}}
    \]
    by induction hypothesis, so that we can define
    \[
      \densem{\Gamma} \xrightarrow{\densem{\pcffix_\sigma\,M}} \densem{\sigma}
    \]
    as the composition
    \(\densem{\Gamma} \xrightarrow{\densem{M}} \densem{\sigma}^{\densem{\sigma}}
    \xrightarrow{\mu} \densem{\sigma}\), where we recall \(\mu\), which assigns
    least fixed points, from~\cref{least-fixed-point-is-continuous}.
  \item
    The interpretation of
    \[
      \def\fCenter{\ \vdash\ }
      \AxiomC{}
      \UnaryInf$\Gamma \fCenter \pcfzero : \pcfnat$
      \DisplayProof
    \]
    is
    \[
      \densem{\Gamma} \xrightarrow{\densem{\pcfzero}} \Nat_\bot
    \]
    which is given by \(\gamma \in \densem{\Gamma} \mapsto 0 \in \Nat_\bot\).
  \item\label{def:interpretation-succ}
    To interpret
    \[
      \def\fCenter{\ \vdash\ }
      \Axiom$\Gamma \fCenter M : \pcfnat$
      \UnaryInf$\Gamma \fCenter \pcfsuc(M) : \pcfnat$
      \DisplayProof
    \]
    we first remark that we have
    \[
      \densem{\Gamma} \xrightarrow{\densem{M}} \Nat_\bot
    \]
    by induction hypothesis, so that we can define
    \[
      \densem{\Gamma} \xrightarrow{\densem{\pcfsuc\,M}} \Nat_\bot
    \]
    as the composition
    \(\densem{\Gamma} \xrightarrow{\densem{M}} \Nat_\bot \xrightarrow{s}
    \Nat_\bot\), where \(s : \Nat_\bot \to \Nat_\bot\) is the successor function
    on \(\Nat_\bot\), i.e.\ \(s(\bot) \coloneqq \bot\) and
    \(s(n) \coloneqq n+1\).
  \item\label{def:interpretation-pred}
    To interpret
    \[
      \def\fCenter{\ \vdash\ }
      \Axiom$\Gamma \fCenter M : \pcfnat$
      \UnaryInf$\Gamma \fCenter \pcfpred(M) : \pcfnat$
      \DisplayProof
    \]
    we first remark that we have
    \[
      \densem{\Gamma} \xrightarrow{\densem{M}} \Nat_\bot
    \]
    by induction hypothesis, so that we can define
    \[
      \densem{\Gamma} \xrightarrow{\densem{\pcfpred\,M}} \Nat_\bot
    \]
    as the composition
    \(\densem{\Gamma} \xrightarrow{\densem{M}} \Nat_\bot \xrightarrow{p}
    \Nat_\bot\), where \(p : \Nat_\bot \to \Nat_\bot\) is the predecessor
    function on \(\Nat_\bot\), i.e.\ \(p(\bot) \coloneqq \bot\),
    \(p(0) \coloneqq 0\) and \(p(n+1) \coloneqq n\).
  \item\label{def:interpretation-ifzero}
    Finally, to interpret
    \[
      \def\fCenter{\ \vdash\ }
      \Axiom$\Gamma \fCenter M : \pcfnat$
      \Axiom$\Gamma \fCenter N_1 : \pcfnat$
      \Axiom$\Gamma \fCenter N_2 : \pcfnat$
      \TrinaryInf$\Gamma \fCenter \pcfifz(M,N_1,N_2) : \pcfnat$
      \DisplayProof
    \]
    we first remark that we have
    \[
      \densem{\Gamma} \xrightarrow{\densem{M}} \Nat_\bot,
      \densem{\Gamma} \xrightarrow{\densem{N_1}} \Nat_\bot \text{ and }
      \densem{\Gamma} \xrightarrow{\densem{N_2}} \Nat_\bot
    \]
    by induction hypothesis, so that we can define
    \[
      \densem{\Gamma} \xrightarrow{\densem{\pcfifz(M,N_1,N_2)}} \Nat_\bot
    \]
    as the composition
    \(\densem{\Gamma}
    \xrightarrow{\langle{\densem{M},\densem{N_1},\densem{N_2}}\rangle}
    {{\Nat_\bot} \times {\Nat_\bot} \times {\Nat_\bot}} \xrightarrow{c}
    \Nat_\bot\), where
    \[
      c : \Nat_\bot \times \Nat_\bot \times \Nat_\bot \to \Nat_\bot
    \] is defined as the \(\omega\)-continuous function
    \(c(\bot,y,z) \coloneqq \bot\), \(c(0,y,z) \coloneqq y\) and
    \(c(n+1,y,z) \coloneqq z\). \qedhere
  \end{enumerate}
\end{definition}

\begin{remark}[Programs of type \(\sigma\) are interpreted as elements of
  \(\densem{\sigma}\)]
  Note that a program, i.e.\ a closed term, \({} \vdash M : \sigma\) is
  interpreted as a function
  \(\densem{M} \colon \One \to \densem{\sigma}\).
  %
  Since the underlying set of \(\One\) is the singleton \(\set{\star}\), this
  function simply picks out an element. % of \(\densem{\sigma}\).
  %
  With this in mind, we see that programs of type \(\sigma\) are interpreted
  as elements of \(\densem{\sigma}\).
\end{remark}


\begin{exercise}\label{exer:interpretation-of-numerals}
  Show that the numerals of PCF are interpreted as the natural numbers in
  \(\Nat_\bot\), i.e.\ \(\densem{\numeral{n}} = n \in \Nat_\bot\) for every
  natural number \(n \in \Nat\).
\end{exercise}

\section{Towards soundness}

The next section will be devoted to proving the \emph{soundness} theorem, which
says that if \(M \bigstep N\), then \(\densem{M} = \densem{N}\). In other words,
the interpretation respects the operational semantics.
%
The proof of soundness relies on the following lemma:% , which is actually a
% particular case of the fact that \(M \smallstep N\) implies
% \(\densem{M} = \densem{N}\).

\begin{lemma}[\(\beta\)-equality]\label{beta-equality}
  The Scott model of PCF validates \(\beta\)-equality. That is, if
  \(\Gamma,\var x : \sigma \vdash M : \tau\) and \(\Gamma \vdash N : \sigma\),
  then
  \[
    \densem*{\pa*{\lambdadot{\var x : \sigma}{M}}\,N} = \densem{M[N/\var x]}
  \]
\end{lemma}

\begin{exercise}\label{exer:beta-equality}
  Prove~\cref{beta-equality} from the substitution lemma given below.
\end{exercise}


Suppose that we have a term \(M : \tau\) in context
\(\Gamma = [\var x_0 : \sigma_0 , \dots , \var x_{k-1} : \sigma_{k-1}]\), and
assume that we have another context \(\Delta\) with terms
\(\Delta \vdash N_i : \sigma_i\) for every \(0 \leq i \leq k-1\).
%
This yields (by a recursive definition on the structure of derivations of
\(\Gamma \vdash M : \tau\)) a term
\[
  \Delta \vdash M[N_0/\var x_0,\dots,N_{k-1}/\var x_{k-1}] : \tau
\]
in context \(\Delta\).

Thus, in our interpretation, we get
\begin{equation*}\label{subst-1}\tag{\(\dagger\)}
  \densem{M[N_0/\var x_0,\dots,N_{k-1}/\var x_{k-1}]}(\delta) \in \densem{\tau}
\end{equation*}
for every \(\delta \in \densem{\Delta}\).

Alternatively, we can note that \(\densem{N_i}(\delta) \in \densem{\sigma_i}\)
for every \(0 \leq i \leq k-1\), so that we can feed them as inputs to
\(\densem{M} \colon \densem{\sigma_0} \times \cdots \times \densem{\sigma_{k-1}}
\to \densem{\tau}\) which yields:
\begin{equation*}\label{subst-2}\tag{\(\ddagger\)}
  \densem{M}\pa*{\densem{N_0}(\delta),\dots,\densem{N_{k-1}}(\delta)} \in \densem{\tau}.
\end{equation*}

The substitution lemma says that \eqref{subst-1} and \eqref{subst-2} agree.

\begin{lemma}[Substitution lemma]\label{substitution-lemma}
  If
  \([\var x_0 : \sigma_0,\dots,\var x_{k-1} : \sigma_{k-1}] = \Gamma \vdash M :
  \tau\), then for every context \(\Delta\) and terms
  \(\Delta \vdash N_i : \sigma_i\) with \(0 \leq i \leq k-1\), we have
  \[
    \densem{M[N_0/\var x_0,\dots,N_{k-1}/\var x_{k-1}]}(\delta)
    =
    \densem{M}\pa*{\densem{N_0}(\delta),\dots,\densem{N_{k-1}}(\delta)}
  \]
  for every \(\delta \in \densem{\Delta}\).
\end{lemma}
\begin{proof}
  By induction on the structure of derivations of \(\Gamma \vdash M : \tau\).

  For example, for the case of \(\lambda\)-abstraction, we consider the term
  \(\Gamma,y : \tau \vdash M : \rho\) with
  \([\var x_0 : \sigma_0,\dots,\var x_{k-1} : \sigma_{k-1}] = \Gamma\), and we
  assume to have a context \(\Delta\) with terms
  \(\Delta \vdash N_i : \sigma_i\) for \(0 \leq i \leq k-1\).
  We have to prove that
  \[
    \densem*{\pa*{\lambdadot{\var y : \tau}{M}}[N_0/\var x_0 , \dots ,
      N_{k-1}/\var x_{k-1}]}(\delta) = \densem*{\lambdadot{\var y :
        \tau}{M}}\pa*{\densem{N_0}(\delta),\dots,\densem{N_{k-1}}(\delta)}
  \]
  for every \(\delta \in \densem{\Delta}\).

  Note that this is an equality of elements of
  \(\densem{\rho}^{\densem{\tau}}\), i.e.\ an equality of
  (\(\omega\)-continuous) functions, so let \(t \in \densem{\tau}\) be
  arbitrary and note that
  \begin{align*}
    &\hspace{13.5pt} \pa*{\densem*{\pa*{\lambdadot{\var y : \tau}{M}}[N_0/\var x_0 , \dots ,
    N_{k-1}/\var x_{k-1}]}(\delta)}(t) \\
    &= \densem*{M[N_0/\var x_0,\dots,N_{k-1}/\var x_{k-1},\var y/\var y]}(\delta,t) \\
    &= \densem*{M}\pa*{\densem{N_0}(\delta,t),\dots,\densem{N_{k-1}}(\delta,t),\densem{\var y}(\delta,t)}
    &\text{(by IH applied to \(\Gamma,\var y : \tau \vdash M : \rho\))} \\
    &= \densem*{M}\pa*{\densem{N_0}(\delta,t),\dots,\densem{N_{k-1}}(\delta,t),t} \\
    &= \pa*{\densem*{\lambdadot{\var y :
        \tau}{M}}\pa*{\densem{N_0}(\delta),\dots,\densem{N_{k-1}}(\delta)}}(t),
  \end{align*}
  as desired.
  %
  Note that the \(N_i\) in lines 2--4 refer to the terms \(N_i\) in the extended
  context \(\Delta,\var y : \tau\).
\end{proof}

\section{Soundness}\label{sec:soundness}

The soundness theorem expresses that the denotational semantics, in the form of
the Scott model of PCF, respects the operational semantics.

\begin{theorem}[Soundness]\label{soundness}
  For terms \(M\) and \(N\) of the same type in the same context, we have: if
  \(M \bigstep N\), then \(\densem{M} = \densem{N}\).
\end{theorem}
\begin{proof}
  By induction on the structure of the derivations of \(M \bigstep N\).
  %
  We work out two cases: the fixed point operator and application.

  \begin{itemize}
  \item Recall that the former is the rule
  \[
    \AxiomC{\(M(\pcffix_{\sigma} \, M) \bigstep V\)}
    \UnaryInfC{\({\pcffix_{\sigma} \, M} \bigstep V\)}
    \DisplayProof
  \]
  so that we may assume \(\densem{M(\pcffix_\sigma\,M)} = \densem{V}\) and we
  have to prove:
  \[
    \densem{\pcffix_{\sigma} \, M} = \densem{V}.
  \]
  But this holds: if \(M\) is a term in context \(\Gamma\), then for
  every \(\gamma \in \densem{\Gamma}\), we have
  \begin{align*}
    \densem{\pcffix_{\sigma} \,M}(\gamma)
    &= \mu\pa*{\densem{M}(\gamma)}
    &\text{(by definition)} \\
    &= \pa*{\densem{M}(\gamma)}\pa*{\mu\pa*{\densem{M}(\gamma)}}
    &\text{(because \(\mu\) gives a fixed point)} \\
    &= \densem{M\pa*{\pcffix_{\sigma}\,M}}(\gamma)
    &\text{(by definition)} \\
    &= \densem{V}(\gamma) &\text{(by assumption)}.
  \end{align*}
  \item   The big-step rule for application is
  \[
    \AxiomC{\(M \bigstep \lambdadot{\var x : \sigma}{E}\)}
    \AxiomC{\(E[N/x] \bigstep V\)}
    \BinaryInfC{\(M \, N \bigstep V\)}
    \DisplayProof
  \]
  so that we may assume
  \(\densem{M} = \densem{\lambdadot{\var x : \sigma}{E}}\)
  and
  \(\densem{E[N/x]} = \densem{V}\),
  and we have to prove:
  \[
    \densem{M \, N} = \densem{V}.
  \]
  If \(M\) and \(N\) are terms in a context \(\Gamma\), then for every
  \(\gamma \in \densem{\Gamma}\), we calculate:
  \begin{align*}
    \densem{M \, N}(\gamma)
    &= \densem{M}(\gamma)\pa*{\densem{N}(\gamma)}
    &\text{(by definition)} \\
    &= \densem{\lambdadot{\var x : \sigma}{E}}(\gamma)\pa*{\densem{N}(\gamma)}
    &\text{(by assumption on \(\densem{M}\))} \\
    &= \densem*{\pa*{\lambdadot{\var x : \sigma}{E}}N}(\gamma)
    &\text{(by definition)} \\
    &= \densem{E[N/\var x]}(\gamma)
    &\text{(by \cref{beta-equality})} \\
    &= \densem{V}(\gamma)
    &\text{(by assumption on \(\densem{N}\))},
  \end{align*}
  completing the proof. \qedhere
  \end{itemize}
\end{proof}

\section{List of exercises}
\begin{enumerate}
\item \cref{exer:interpretation-of-numerals}: On the interpretation of the numerals.
\item \cref{exer:beta-equality}: On proving that the interpretation validates
  \(\beta\)-equality.
\end{enumerate}

%%% Local Variables:
%%% mode: latexmk
%%% TeX-master: "../main"
%%% End:

\chapter{Computational adequacy}\label{chap:comp-adequacy}

Soundness is a nice and fundamental result, but a converse would be more
interesting: Can we use the model to compute in PCF? More precisely, if two
terms \(M\) and \(N\) are equal in the model, does \(M\) reduce to \(N\) (or
vice versa) in the operational semantics?

This is actually not the case, because the model is more extensional than PCF,
e.g.\ the programs \(\lambdadot{\var x : \pcfnat}{\var x}\) and
\(\lambdadot{\var x : \pcfnat}{\pcfpred\pa*{\pcfsuc\,\var x}}\) are different
values in PCF, but their interpretations as \(\omega\)-continuous functions are
equal.

However, it \emph{is} the case that if \(M\) is a program of type \(\pcfnat\)
and \(\densem{M} = n\), then \(M \bigstep \numeral{n}\).
%
This is known as the \emph{computational adequacy} of the Scott model of PCF and
allows us to compute in PCF using the domain-theoretic denotational semantics.

\begin{theorem*}[Computational adequacy]\label{adequacy}
  For every program \(M\) of type \(\pcfnat\) and natural number \(n \in \Nat\),
  if \(\densem{M} = n\), then \(M \bigstep \numeral n\).
\end{theorem*}

Unlike soundness, computational adequacy cannot be proved by a straightforward
induction on types, or structure of the derivation of the program \(M\), since
the statement refers to closed terms of type \(\pcfnat\) only.

Therefore, instead of proving computational adequacy directly, we will derive it
as a corollary of a more general result that does allow for a proof by
induction on types.

More specifically, we will introduce a \emph{logical
  relation}~\cite{Plotkin1973}. It is an example of a fundamental and often used
technique in the theory of programming languages, going back to \emph{Tait's
  method of computability}~\cite{Tait1967} and also known as the method of
\emph{reducibility candidates}~\cite{Girard1989}.

\section{The logical relation}

We introduce the logical relation \(R_\sigma\), a binary relation between
semantics and syntax of PCF, and use it to prove computational adequacy.

\begin{definition}[Logical relation, \(\logrel_\sigma\)]\label{def:logical-relation}
  We define a binary relation \(\logrel_\sigma\) between elements of
  \(\densem{\sigma}\) and programs of type \(\sigma\) by induction on types:
  \begin{enumerate}[(i)]
  \item\label{R-nat} \({x \logrel_{\pcfnat} M}\) holds if \(x = n\) with
    \(n \in \Nat\) implies \(M \bigstep \numeral n\),
  \item\label{R-fun} \(f \logrel_{\sigma \pcffun \tau} M\) holds if \(x \logrel_\sigma N\)
    implies \({f(x)} \logrel_\tau {M \, N}\) for every element
    \(x \in \densem{\sigma}\) and program \(N\) of type \(\sigma\). \qedhere
  \end{enumerate}
\end{definition}

Observe that computational adequacy is equivalent to the statement that for
every program \(M : \pcfnat\) we have \(\densem{M} \logrel_\pcfnat M\).
%
Hence, we will have proven computational adequacy if we can show:

\begin{lemma*}
  For every program \(M : \sigma\), it holds that \(\densem{M} \logrel_\sigma M\).
\end{lemma*}

This, in turn, will follow from the fundamental theorem of the logical relation:

\begin{lemma*}[Fundamental theorem of the logical relation]\label{fundamental-theorem}
  If
  \([\var x_0 : \sigma_0 , \dots , \var x_{k-1} : \sigma_{k-1}] = \Gamma \vdash
  M : \tau\), then whenever we have \(\Delta \vdash N_i : \sigma_i\) and
  \(s_i \in \densem{\sigma_i}\) such that \(s_i \logrel_{\sigma_i} N_i\) for
  all \(0 \leq i \leq k-1\), it holds that
  \[
    \pa*{\densem{M}\pa*{s_0,\dots,s_{k-1}}} \logrel_{\tau} {M[N_0/\var x_0,\dots,N_{k-1}/\var x_{k-1}]}.
  \]
\end{lemma*}

The second clause, item~\ref{R-fun}, of \cref{def:logical-relation} says that
related inputs must go to related outputs and is designed to make proofs by
induction on types possible.

Before we can prove the fundamental theorem of the logical relation, we need to
establish several properties of the relation \(\logrel_\sigma\).

\begin{lemma}\label{R-extend-left}
  If \(x \below y\) and \(y \logrel_\sigma M\), then \(x \logrel_\sigma M\).
\end{lemma}
\begin{proof}
  We proceed by induction on the type \(\sigma\).
  %
  For the base case, suppose we have \(x \below y\) in \(\Nat_\bot\) and
  \(y \logrel_\pcfnat M\), and assume that \(x = n\) for a natural number
  \(n\). We have to prove that \(M \bigstep \numeral n\).
  %
  Since \(n = x \below y\), we must have \(y = n\) by definition of the partial
  order on \(\Nat_\bot\), and hence \(M \bigstep \numeral n\) because we assumed
  \(y \logrel_\pcfnat M\).

  For function types, we assume to have \(f \below g\) in
  \(\densem{\tau}^{\densem{\sigma}}\) with
  \(g \logrel_{\sigma \pcffun \tau} M\), and we have to prove
  \(f \logrel_{\sigma \pcffun \tau} M\).
  %
  So suppose that we have \(x \logrel_\sigma N\). Then
  \(g(x) \logrel_\tau M\,N\) because we assumed
  \(g \logrel_{\sigma \pcffun \tau} M\).
  %
  But \(f \below g\), so \(f(x) \below g(x)\) and hence, we get the desired
  \(f(x) \logrel_\tau M\,N\) by the induction hypothesis applied at the type
  \(\tau\).
\end{proof}

\begin{lemma}\label{contains-bot}
  The least element is related to all programs.
  %
  That is, for every program \(M : \sigma\), we have \(\bot \logrel_\sigma M\),
  where we recall that \(\bot\) denotes the least element of
  \(\densem{\sigma}\).
\end{lemma}
\begin{lemma}\label{closure-under-omega-chains}
  The logical relation is closed under least upper bounds of \(\omega\)-chains.
  %
  That is, for every program \(M : \sigma\) and \(\omega\)-chain
  \(x_0 \below x_1 \dots\) in \(\densem{\sigma}\), if \(x_n \logrel_\sigma M\)
  for every \(n \in \Nat\), then \(\pa*{\bigsqcup_{n \in \Nat}x_n} \logrel_\sigma M\).
\end{lemma}

\begin{exercise}\label{exer:contains-bot-and-closure-under-omega-chains}
  Prove~\cref{contains-bot,closure-under-omega-chains} by induction on PCF
  types.
\end{exercise}

\begin{exercise}\label{exer:closure-under-basic-operations}
  Check the following closure properties of \(\logrel_\pcfnat\):
  \begin{enumerate}[(i)]
  \item \(0 \logrel_\pcfnat \numeral{0}\);
  \item if \(x \logrel_\pcfnat M\), then \(s(x) \logrel_\pcfnat \pcfsuc\, M\),
    where \(s\) is the successor map on \(\Nat_\bot\) as
    in~\cref{def:interpretation}\ref{def:interpretation-succ};
  \item if \(x \logrel_\pcfnat M\), then \(p(x) \logrel_\pcfnat \pcfpred\, M\),
    where \(p\) is the predecessor map on \(\Nat_\bot\) as
    in~\cref{def:interpretation}\ref{def:interpretation-pred};
  \item if \(x \logrel_\pcfnat M\), \(y \logrel_\pcfnat N_1\) and
    \(z \logrel_\pcfnat N_2\), then
    \(c(x,y,z) \logrel_\pcfnat \pcfifz(M,N_1,N_2)\), where \(c\) is the if-zero
    map on \(\Nat_\bot\) as
    in~\cref{def:interpretation}\ref{def:interpretation-ifzero}. \qedhere
  \end{enumerate}
\end{exercise}

\section{Applicative approximation}

We will need one more ingredient for proving the fundamental theorem of the
logical relation and (hence) computational adequacy: the \emph{applicative
  approximation} relation.

In~\cref{R-extend-left}, we showed that if \(x \below y\) and
\(y \logrel_\sigma M\), then \(x \logrel_\sigma M\).
%
The applicative approximation relation may be motivated by the desire to have a
similar property of the logical relation, but for programs, rather than elements
of the interpretation. That is, we introduce a binary relation
\(\appapprox_\sigma\) on programs of type \(\sigma\) and show
(\cref{R-extend-right}) that if \(x \logrel_\sigma M\) and
\(M \appapprox_\sigma N\), then \(x \logrel_\sigma N\).

\begin{definition}[Applicative approximation]
  We define a binary relation \(\appapprox_\sigma\) on programs of type
  \(\sigma\) by induction on types:
  \begin{enumerate}[(i)]
  \item \(M \appapprox_\pcfnat N\) holds if \(M \bigstep \numeral{n}\) implies
    \(N \bigstep \numeral{n}\) for every natural number \(n\in\Nat\).
  \item \(M \appapprox_{\sigma \pcffun \tau} N\) holds if for every program
    \(K\) of type \(\sigma\) we have \(M \, K \appapprox_\tau N \, K\). \qedhere
  \end{enumerate}
\end{definition}

Note that the applicative approximation relation mirrors the partial orders on
\(\Nat_\bot\) and the exponentials.
%
But the applicative approximation relation is only a \emph{preorder}; it is
reflexive and transitive, but not antisymmetric:

\begin{exercise}\label{exer:applicative-approximation-not-antisymmetric}
  Give examples of programs \(M\) and \(N\) such that
  \(M \appapprox_{\pcfnat \pcffun \pcfnat} N\) and
  \(N \appapprox_{\pcfnat \pcffun \pcfnat} M\), but \(M \neq N\).
\end{exercise}

\begin{lemma}\label{R-extend-right}
  If \(x \logrel_\sigma M\) and \(M \appapprox_\sigma N\), then
  \(x \logrel_\sigma N\).
\end{lemma}
\begin{proof}
  We proceed by induction on types.
  %
  For the base type, we assume \(x \logrel_\pcfnat M\) and
  \(M \appapprox_\pcfnat N\). Suppose that \(x = n\) for \(n \in \Nat\); we have
  to prove \(N \bigstep \numeral n\). But \(M \bigstep \numeral n\) since
  \(x \logrel_\pcfnat M\) and hence, \(N \bigstep \numeral n\) because
  \(M \appapprox_\pcfnat N\).

  For function types, we assume \(f \logrel_{\sigma \pcffun \tau} M\) and
  \(M \appapprox_{\sigma \pcffun \tau} N\). Suppose that we have
  \(x \logrel_\sigma K\); we have to prove \(f(x) \logrel_\tau N \, K\).
  %
  By our assumption that \(f \logrel_{\sigma \pcffun \tau} M\), we get
  \(f(x) \logrel_\tau M \, K\). Moreover, our assumption
  \(M \appapprox_{\sigma \pcffun \tau} N\) yields
  \(M \, K \appapprox_\tau N \, K\), so that \(f(x) \logrel_\tau N \, K\) by the
  induction hypothesis applied at \(\tau\), as desired.
\end{proof}

The following lemma has two useful corollaries for proving the fundamental
theorem of the logical relation:
\begin{lemma}\label{applicative-approximation-if-big-step-implication}
  For all programs \(M\) and \(N\) of type \(\sigma\), if \(M \bigstep V\)
  implies \(N \bigstep V\) for every program \(V : \sigma\), then
  \(M \appapprox_\sigma N\).
\end{lemma}

\begin{exercise}\label{exer:applicative-approximation-if-big-step-implication}
  Prove~\cref{applicative-approximation-if-big-step-implication} by induction on
  types.
\end{exercise}

\begin{corollary}\label{applicative-approximation-fix}
  For every program \(M : \sigma \pcffun \sigma\), it holds
  that \(M\pa*{\pcffix_\sigma \, M} \appapprox_\sigma \pcffix_\sigma M\).
\end{corollary}
\begin{proof}
  By \cref{exer:applicative-approximation-if-big-step-implication} it is enough
  to establish that \(M\pa*{\pcffix_\sigma\,M} \bigstep V\) implies
  \(\pcffix_\sigma \, M \bigstep V\) for every program \(V\).
  %
  But this holds by definition of the big-step operational semantics.
\end{proof}

\begin{corollary}\label{applicative-approximation-beta}
  For every term \(\var x : \sigma \vdash M : \tau\) and program \(N : \sigma\),
  it holds that \(M[N/\var x] \appapprox_\tau \pa*{\lambdadot{\var x : \sigma}{M}}N\).
\end{corollary}
\begin{proof}
  By~\cref{applicative-approximation-if-big-step-implication} and the definition
  of the big-step operational semantics.
\end{proof}

\cref{applicative-approximation-fix} allows us to prove that the logical
relation is closed under the least fixed point operation \(\mu\)
(recall~\cref{least-fixed-point}):

\begin{lemma}\label{closure-under-mu}
  If \(f \logrel_{\sigma \pcffun \sigma} M\), then \(\mu(f) \logrel_\sigma {\pcffix_\sigma\,M}\).
\end{lemma}
\begin{proof}
  Suppose that \(f \logrel_{\sigma \pcffun \sigma} M\). Since
  \(\mu(f) = \bigsqcup_{n \in \Nat}f^n(\bot)\),
  \cref{closure-under-omega-chains} tells us that it is enough to prove that
  \(f^n(\bot) \logrel_\sigma \pa*{\pcffix_\sigma\,M}\) for every \(n \in \Nat\).
  %
  We do so by induction on \(n\). The case \(n = 0\) is provided
  by~\cref{contains-bot}. For the inductive case, suppose that
  \(f^n(\bot) \logrel_\sigma \pa*{\pcffix_\sigma\,M}\); we show that
  \(f^{n+1}(\bot) \logrel_\sigma \pa*{\pcffix_\sigma\,M}\).
  %
  By induction hypothesis and the assumption that
  \(f \logrel_{\sigma \pcffun \sigma} M\), we see that
  \(f\pa*{f^n(\bot)} \logrel_\sigma M\,\pa*{\pcffix_\sigma \, M}\).
  %
  Hence, we obtain the desired
  \(f^{n+1}(\bot) \logrel_\sigma \pa*{\pcffix_\sigma\,M}\)
  by \cref{R-extend-right,applicative-approximation-fix}.
\end{proof}

\section{Proving computational adequacy}

Finally, we have established enough properties to prove:
\begin{lemma}[Fundamental theorem of the logical relation]\label{fundamental-theorem}
  If \(M : \tau\) is a term in a context
  \(\Gamma = [\var x_0 : \sigma_0 , \dots , \var x_{k-1} : \sigma_{k-1}]\), then
  whenever we have \(\Delta \vdash N_i : \sigma_i\) and
  \(s_i \in \densem{\sigma_i}\) such that \(s_i \logrel_{\sigma_i} N_i\) for all
  \(0 \leq i \leq k-1\), it holds that
  \[
    \pa*{\densem{M}\pa*{s_0,\dots,s_{k-1}}} \logrel_{\tau} {M[N_0/\var x_0,\dots,N_{k-1}/\var x_{k-1}]}.
  \]
\end{lemma}

\begin{proof}
  We abbreviate \(s_0,\dots,s_{k-1}\) as \(\vec s\) and
  \(M[N_0/\var x_0,\dots,N_{k-1}/\var x_{k-1}]\) as
  \(M[\vec{N}/\vec{\var{x}}]\). We proceed by induction on the structure of the
  derivations \(\Gamma \vdash M : \tau\):

  \begin{itemize}
  \item For
    \([\var x_0 : \sigma_0,\dots,\var x_{k-1} : \sigma_{k-1}] \vdash \var x_i :
    \sigma_i\) and \(s_i \logrel_{\sigma_i} N_i\) for all
    \(0 \leq i \leq k-1\), we observe that
    \[ {\densem{\var x_i}(\vec s)} = {s_i} \logrel_{\sigma_i} {N_i} =
      {{\var x_i}[\vec N/\vec{\var{x}}]},
    \]
    as wished.
  \item For application, we consider \(\Gamma \vdash M : \tau \pcffun \rho\) and
    \(\Gamma \vdash K : \tau\) and we have to prove
    \[
      \densem{M \, K}(\vec s) \logrel_\rho \pa*{M \, K}[\vec N/\vec{\var x}].
    \]
    Our induction hypothesis yields
    \[
      \densem{M}(\vec s) \logrel_{\tau \pcffun \rho} M[\vec N/\vec{\var{x}}]
      \quad\text{and}\quad
      \densem{K}(\vec s) \logrel_\tau K[\vec N/\vec{\var x}].
    \]
    And hence, by definition of \(\logrel_{\tau \pcffun \rho}\),
    \[
      {\densem{M}(\vec s)\pa*{\densem{K}(\vec s)}} \logrel_\rho
      M[\vec N/\vec{\var x}] \, K[\vec N/\vec {\var x}],
    \]
    but this is an unfolded version of what we wanted to prove.
  \item For \(\lambda\)-abstraction, we consider
    \(\Gamma,\var y : \tau \vdash M : \rho\) and we have to prove
    \[
      \densem{\lambdadot{\var y : \tau}{M}}(\vec s) \logrel_{\tau \pcffun \rho}
      \pa*{\lambdadot{\var y: \tau}{M}}[\vec{N}/\vec{\var{x}}].
    \]
    So suppose that we have \(t \logrel_\tau K\); we show that
    \[
      \pa*{\densem{\lambdadot{\var y : \tau}{M}}(\vec s)}(t) \logrel_{\rho}
      \pa[\big]{\pa*{\lambdadot{\var y: \tau}{M}}[\vec{N}/\vec{\var{x}}]}K.
    \]
    The left-hand side unfolds to \(\densem{M}(\vec s,t)\), while the right-hand
    side unfolds to
    \(\pa[\big]{\lambdadot{\var y : \tau}{M[\vec{N}/\vec{\var{x}}]}}K\).
    By~\cref{R-extend-right,applicative-approximation-beta}, it thus suffices to
    prove
    \[
      \densem{M}(\vec s,t) \logrel_\rho M[\vec{N}/\vec{\var x},K/\var y],
    \]
    which holds by induction hypothesis.
  \item
    For \(\pcffix\), we consider \(\Gamma \vdash M : \tau \pcffun \tau\) and we have to prove
    \[
      \densem{\pcffix_\tau\,M}(\vec s) \logrel_\tau
      \pa*{\pcffix_\tau\,M}[\vec{N}/\vec{\var x}].
    \]
    By definition,
    \(\densem{\pcffix_\tau\,M}(\vec s) = \mu\pa*{\densem{M}(\vec s)}\) and
    \(\pa*{\pcffix_\tau\, M}[\vec{N}/\vec{\var x}] =
    \pcffix_\tau\pa{M[\vec{N}/\vec{\var{x}}]}\), so by~\cref{closure-under-mu},
    it is enough to prove that
    \(\densem{M}(\vec s) \logrel_{\tau \pcffun \tau}
    M[\vec{N}/\vec{\var{x}}]\). But this holds by induction hypothesis.
  \item Finally, the cases for \(\pcfzero\), \(\pcfsuc\), \(\pcfzero\) and
    \(\pcfifz\) follow from~\cref{exer:closure-under-basic-operations}. \qedhere
  \end{itemize}
\end{proof}

For the special case of the empty context, we obtain:
\begin{corollary}
  For every program \(M : \sigma\), it holds that
  \(\densem{M} \logrel_\sigma M\).
\end{corollary}
%
By further specialising to the base type and the definition of
\(\logrel_\pcfnat\), we get:
\begin{theorem}[Computational adequacy]\label{adequacy}
  For every program \(M\) of type \(\pcfnat\) and natural number \(n \in \Nat\),
  if \(\densem{M} = n\), then \(M \bigstep \numeral n\).
\end{theorem}

We end these notes with an exercise illustrating two uses of computational
adequacy: one on establishing the computational behaviour of a program, and one
on using a domain-theoretic argument to show that a certain program cannot
exist.

\begin{exercise}\label{exer:finding-roots}
  Consider the following pseudocode:

\begin{verbatim}
  root : (Nat -> Nat) -> Nat
  root M = root' M 0

  root' : (Nat -> Nat) -> Nat -> Nat
  root' M m = if   M 0 == 0
              then m
              else root' (shift M) (m + 1)

  shift : (Nat -> Nat) -> Nat -> Nat
  shift M n = M (n + 1)
\end{verbatim}
  \begin{enumerate}[(i), itemsep=5pt]
  \item Translate the above pseudocode to corresponding programs \(\pcfroot\),
    \(\pcfrootprime\) and \(\pcfshift\) in PCF.
  \item Use computational adequacy to prove that for every program
    \(M : \pcfnat \pcffun \pcfnat\) and natural number \(n\), we have
    \[
      \pcfroot\,M \bigstep \numeral{n} \iff %
      M \, \numeral{n} \bigstep \numeral{0} \text{ and } %
      \forall_{k < n}\pa*{M \, \numeral{k} \text{ reduces to a non-zero numeral}}.
     \]
     That is, \(\pcfroot \, M\) computes the first ``root'' (or zero) of \(M\)
     provided that \(M\) terminates on all values up to (and including) the
     root.
   \item Give an informal argument as to why we can't drop the condition that
     \(M \, \numeral{k}\) reduces to a (non-zero) numeral for all \(k \leq n\).
   \item Show that there is \emph{no} program
     \(F : (\pcfnat \pcffun \pcfnat) \pcffun \pcfnat\) such that
     for every program \(M : \pcfnat \pcffun \pcfnat\) and natural number \(n\),
     we have
     \[
       F\,M \bigstep \numeral{n} \iff %
       n \text{ is the least natural number with } M \, \numeral{n} \bigstep \numeral{0}.
     \]
     \emph{Hint}: Consider the \(\omega\)-continuous functions
     \(f,g : \Nat_\bot \to \Nat_\bot \) given by \(f(\bot) \coloneqq \bot\),
     \(f(0) \coloneqq \bot\) and \(f(n+1) \coloneqq 0\), and \(g(x) = 0\) for
     all \(x \in \Nat_\bot\).
     %
     Use the observation that \(f \below g\) to derive a contradiction from the
     assumption that a program~\(F\) with the above specification exists.
     %
     \qedhere
  \end{enumerate}
\end{exercise}

\section{List of exercises}
\begin{enumerate}
\item \cref{exer:contains-bot-and-closure-under-omega-chains}: On proving that
  the logical relation contains the least element is and is closed under least
  upper bounds of \(\omega\)-chains.
\item \cref{exer:closure-under-basic-operations}: On showing that the logical
  relation is suitably closed under the basic operations on the type of natural
  numbers.
\item \cref{exer:applicative-approximation-not-antisymmetric}: On a
  counterexample to antisymmetry of the applicative approximation relation.
\item \cref{exer:applicative-approximation-if-big-step-implication}: On a
  sufficient big-step condition for the applicative approximation relation to
  hold.
\item \cref{exer:finding-roots}: On computational adequacy and programs finding roots.
\end{enumerate}

%%% Local Variables:
%%% mode: latexmk
%%% TeX-master: "../main"
%%% End:


\backmatter%
\printbibliography[heading=bibintoc]%

\end{document}
