\chapter{Introduction}

Denotational semantics~\cite{Scott1970,ScottStrachey1971}

% Denotational semantics aims to understand computer programmes by assigning
% mathematical meaning to the syntax of a programming language.
% %
% In this course we will study a simple functional programming language called
% PCF. Notably, this language has general recursion through a fixed point
% operator.
% %
% This means a simple denotational semantics based on sets is not
% suitable. Instead, we interpret the types of PCF as certain partially ordered
% sets leading to domain theory and Scott's model of PCF in particular.
% %
% The central theorems of soundness and computational adequacy, formulated and
% proved by Plotkin, then tell us that a PCF programme computes to a value if and
% only if their interpretations in the model are equal.

Following~\cite{Escardo2007}, we might summarise as:

\begin{displayquote}
  Operational semantics is about \emph{how} we compute.

  Denotational semantics is about \emph{what} we compute.
\end{displayquote}

\section{Aims}

\section{References}

\cite{AbramskyJung1994,GierzEtAl2003,Streicher2006,Hart2020,Escardo2007}

\cite{PittsWinskellFiore2012} lecture notes

\cite{Winskel1993} textbook

\cite{Streicher2006,Gunther1992}


%%% Local Variables:
%%% mode: latexmk
%%% TeX-master: "../main"
%%% End:
