\chapter{The Scott model of PCF}

We started these notes by introducing PCF and its operational semantics
in~\cref{chap:PCF}. We then developed sufficient domain theory
in~\cref{chap:domains} to give a denotational semantics of PCF, where we
interpret the types of PCF as \(\omega\)-cppos and terms as
\(\omega\)-continuous maps. This chapter is devoted to precisely defining that
interpretation, known as the Scott model of PCF.

\begin{definition}[Interpretation of PCF types]
  We inductively assign an \(\omega\)-cppo \(\densem{\sigma}\) to each type
  \(\sigma\) of PCF:
  \begin{alignat*}{3}
    &\densem{\pcfnat} &&\coloneqq \Nat_\bot, \text{and} \\
    &\densem{\sigma \pcffun \tau} &&\coloneqq \densem{\tau}^{\densem{\sigma}},
  \end{alignat*}
  where we recall \(\Nat_\bot\) from~\cref{exam:N_bot} and the exponential
  from~\cref{def:exponential}.
\end{definition}

This interpretation extends to contexts by using iterated binary products.
\begin{definition}[Interpretation of contexts]
  A context
  \(\Gamma = [\var x_0 : \sigma_0, \dots , \var x_{n-1} :
  \sigma_{n-1}]\)
  is interpreted as
  \[
    \densem{\Gamma} \coloneqq \densem{\sigma_0} \times \cdots \times
    \densem{\sigma_{n-1}}
  \]
  with the convention that the empty context is interpreted as the empty
  product: the one point \(\omega\)-cppo \(\One\) from~\cref{exam:one-point}.
\end{definition}

The interpretation of the terms of PCF proceeds by yet another inductive
definition:

\begin{definition}[Interpretation of PCF terms]
  A term \(\Gamma \vdash M : \sigma\) of type \(\sigma\) in context~\(\Gamma\)
  will be interpreted as an \emph{\(\omega\)-continuous} function
  \[
    \densem{\Gamma} \xrightarrow{\densem{M}} \densem{\sigma}
  \]
  according to the following clauses which mirror~\cref{def:PCF-terms}:
  \begin{enumerate}[(i)]
  \item
    The interpretation of
    \[
      \def\fCenter{\ \vdash\ }
      \AxiomC{\phantom{$\fCenter$}}
      \UnaryInf$\Gamma,\var{x}:\sigma,\Delta \fCenter \var{x} : \sigma$
      \DisplayProof
    \]
    is
    \[
      \densem{\Gamma} \times \densem{\sigma} \times {\densem{\Delta}}
      \xrightarrow{\pi} \densem{\sigma},
    \]
    where \(\pi\) is a suitable composition of projections
    (recall~\cref{projections}), which is \(\omega\)-continuous, because
    \(\omega\)-continuous functions are closed under composition
    (recall~\cref{exer:category-of-cpos}).
  \item
    To interpret
    \[
      \def\fCenter{\ \vdash\ }
      \Axiom$\Gamma , \var{x} : \sigma \fCenter M : \tau$
      \UnaryInf$\Gamma \fCenter (\lambdadot{\var{x} : \sigma}{M}) : \sigma
      \pcffun \tau$ \DisplayProof
    \]
    we first remark that we have
    \[
      \densem{\Gamma} \times \densem{\sigma} \xrightarrow{\densem{M}} \densem{\tau}
    \]
    by induction hypothesis, so that we can define
    \[
      \densem{\Gamma} \xrightarrow{\densem{\lambdadot{\var x : \sigma}{M}}}
      \densem{\tau}^{\densem{\sigma}}
    \]
    as the curried (recall~\cref{exer:curry-is-continuous}) version of
    \(\densem{M}\), i.e.\ for \(\gamma \in \densem{\Gamma}\), we have
    \(\densem{\lambdadot{\var x : \sigma}{M}}(\gamma) \coloneqq x \mapsto
    \densem{M}(\gamma,x)\).
  \item
    To interpret
    \[
      \def\fCenter{\ \vdash\ }
      \Axiom$\Gamma \fCenter M : \sigma \pcffun \tau$
      \Axiom$\Gamma \fCenter N : \sigma$
      \BinaryInf$\Gamma \fCenter M \, N : \tau$
      \DisplayProof
    \]
    we first remark that we have
    \[
      \densem{\Gamma} \xrightarrow{\densem{M}} \densem{\tau}^{\densem{\sigma}}
      \quad\text{and}\quad
      \densem{\Gamma} \xrightarrow{\densem{N}} \densem{\sigma}
    \]
    by induction hypothesis, so that we can define
    \[
      \densem{\Gamma} \xrightarrow{\densem{M \, N}} \densem{\tau}
    \]
    by application: for \(\gamma \in \densem{\Gamma}\), we have
    \(\densem{M \, N}(\gamma) \coloneqq
    \densem{M}(\gamma)\pa*{\densem{N}(\gamma)}\).

    More abstractly, it is the composition of
    \[
      \densem{\Gamma} \xrightarrow{\langle \densem{M} , \densem{N} \rangle}
      \densem{\tau}^{\densem{\sigma}} \times \densem{\sigma}
      \xrightarrow{\text{evaluation}}
      \densem{\tau},
    \]
    where we recall~\cref{exer:product-induced,evaluation-is-continuous}.
  \item
    To interpret
    \[
      \def\fCenter{\ \vdash\ }
      \Axiom$\Gamma \fCenter M : \sigma \pcffun \sigma$
      \UnaryInf$\Gamma \fCenter \pcffix_\sigma(M) : \sigma$ \DisplayProof
    \]
    we first remark that we have
    \[
      \densem{\Gamma} \xrightarrow{\densem{M}} \densem{\sigma}^{\densem{\sigma}}
    \]
    by induction hypothesis, so that we can define
    \[
      \densem{\Gamma} \xrightarrow{\densem{\pcffix_\sigma\,M}} \densem{\sigma}
    \]
    as the composition
    \(\densem{\Gamma} \xrightarrow{\densem{M}} \densem{\sigma}^{\densem{\sigma}}
    \xrightarrow{\mu} \densem{\sigma}\), where we recall \(\mu\), which assigns
    least fixed points, from~\cref{least-fixed-point-is-continuous}.
  \item
    The interpretation of
    \[
      \def\fCenter{\ \vdash\ }
      \AxiomC{}
      \UnaryInf$\Gamma \fCenter \pcfzero : \pcfnat$
      \DisplayProof
    \]
    is
    \[
      \densem{\Gamma} \xrightarrow{\densem{\pcfzero}} \Nat_\bot
    \]
    which is given by \(\gamma \in \densem{\Gamma} \mapsto 0 \in \Nat_\bot\).
  \item
    To interpret
    \[
      \def\fCenter{\ \vdash\ }
      \Axiom$\Gamma \fCenter M : \pcfnat$
      \UnaryInf$\Gamma \fCenter \pcfsuc(M) : \pcfnat$
      \DisplayProof
    \]
    we first remark that we have
    \[
      \densem{\Gamma} \xrightarrow{\densem{M}} \Nat_\bot
    \]
    by induction hypothesis, so that we can define
    \[
      \densem{\Gamma} \xrightarrow{\densem{\pcfsuc\,M}} \Nat_\bot
    \]
    as the composition
    \(\densem{\Gamma} \xrightarrow{\densem{M}} \Nat_\bot \xrightarrow{s}
    \Nat_\bot\), where \(s : \Nat_\bot \to \Nat_\bot\) is the successor function
    on \(\Nat_\bot\), i.e.\ \(s(\bot) \coloneqq \bot\) and
    \(s(n) \coloneqq n+1\).
  \item
    To interpret
    \[
      \def\fCenter{\ \vdash\ }
      \Axiom$\Gamma \fCenter M : \pcfnat$
      \UnaryInf$\Gamma \fCenter \pcfpred(M) : \pcfnat$
      \DisplayProof
    \]
    we first remark that we have
    \[
      \densem{\Gamma} \xrightarrow{\densem{M}} \Nat_\bot
    \]
    by induction hypothesis, so that we can define
    \[
      \densem{\Gamma} \xrightarrow{\densem{\pcfpred\,M}} \Nat_\bot
    \]
    as the composition
    \(\densem{\Gamma} \xrightarrow{\densem{M}} \Nat_\bot \xrightarrow{p}
    \Nat_\bot\), where \(p : \Nat_\bot \to \Nat_\bot\) is the predecessor
    function on \(\Nat_\bot\), i.e.\ \(p(\bot) \coloneqq \bot\),
    \(p(0) \coloneqq 0\) and \(p(n+1) \coloneqq n\).
  \item
    Finally, to interpret
    \[
      \def\fCenter{\ \vdash\ }
      \Axiom$\Gamma \fCenter M : \pcfnat$
      \Axiom$\Gamma \fCenter N_1 : \pcfnat$
      \Axiom$\Gamma \fCenter N_2 : \pcfnat$
      \TrinaryInf$\Gamma \fCenter \pcfifz(M,N_1,N_2) : \pcfnat$
      \DisplayProof
    \]
    we first remark that we have
    \[
      \densem{\Gamma} \xrightarrow{\densem{M}} \Nat_\bot,
      \densem{\Gamma} \xrightarrow{\densem{N_1}} \Nat_\bot \text{ and }
      \densem{\Gamma} \xrightarrow{\densem{N_2}} \Nat_\bot
    \]
    by induction hypothesis, so that we can define
    \[
      \densem{\Gamma} \xrightarrow{\densem{\pcfifz(M,N_1,N_2)}} \Nat_\bot
    \]
    as the composition
    \(\densem{\Gamma}
    \xrightarrow{\langle{\densem{M},\densem{N_1},\densem{N_2}}\rangle}
    {{\Nat_\bot} \times {\Nat_\bot} \times {\Nat_\bot}} \xrightarrow{c}
    \Nat_\bot\), where
    \[
      c : \Nat_\bot \times \Nat_\bot \times \Nat_\bot \to \Nat_\bot
    \] is defined as the \(\omega\)-continuous function
    \(c(\bot,y,z) \coloneqq \bot\), \(c(0,y,z) \coloneqq y\) and
    \(c(n+1,y,z) \coloneqq z\).
  \end{enumerate}
\end{definition}

\begin{remark}[Programs of type \(\sigma\) are interpreted as elements of
  \(\densem{\sigma}\)]
  Note that a program, i.e.\ a closed term, \({} \vdash M : \sigma\) is
  interpreted as a function
  \(\densem{M} \colon \One \to \densem{\sigma}\).
  %
  Since the underlying set of \(\One\) is the singleton \(\set{\star}\), this
  function simply picks out an element of \(\densem{\sigma}\).
  %
  With this in mind, we see that programs of type \(\sigma\) are interpreted
  as elements of \(\densem{\sigma}\).
\end{remark}


\begin{exercise}\label{exer:interpretation-of-numerals}
  Show that the numerals of PCF are interpreted as the natural numbers in
  \(\Nat_\bot\), i.e.\ \(\densem{\numeral{n}} = n \in \Nat_\bot\) for every
  natural number \(n \in \Nat\).
\end{exercise}

%%% Local Variables:
%%% mode: latexmk
%%% TeX-master: "../main"
%%% End:
